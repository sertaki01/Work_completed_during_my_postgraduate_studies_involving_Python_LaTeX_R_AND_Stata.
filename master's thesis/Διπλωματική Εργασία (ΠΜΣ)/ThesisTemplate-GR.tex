\documentclass[12pt,a4paper,oneside]{book}
%%%%%%%%%%%%%%%%%%%%%% LINE SPACING PRIOR TO MAIN TEXT %%%%%%%%%%%%%%%%%%%%%%
\renewcommand{\baselinestretch}{1.5} 

%%%%%%%%%%%%%%%%%%%%%% FONT SELECTION %%%%%%%%%%%%%%%%%%%%%%
\usepackage[greek]{babel}

\usepackage[utf8x]{inputenc}

%%%%%%%%%%%%%%%%%%%%%%%%%%%%%%%%%%%%%%%%%% 
%
% List of various  fonts
% NOTE: Some may required installation
% if not available on your system
%
%%%%%%%%%%%%%%%%%%%%%%%%%%%%%%%%%%%%%%%%%%
 
% \usepackage[default]{gfsneohellenic}
% \usepackage{gfsbaskerville}
% \renewcommand{\familydefault}{\sfdefault}


%  The used font. Avaliable on all tex installations
%  and hence should not pose any problems.
\usepackage{times}



 
%%%%%%%%%%%%%%%%%%%%%% ENGLISH TEXT
\newcommand{\la}{\latintext}
%%%%%%%%%%%%%%%%END OF FONT SELECTION %%%%%%%%%%%%%%%%%%%%%%
\usepackage{amsmath,amsfonts,amssymb,amsthm}
\usepackage{setspace}
\usepackage{url}
\usepackage{graphicx}
\usepackage{caption}
\usepackage{float}
\usepackage[top=1.25in, bottom=1.25in, left=1.6in, right=1.15in]{geometry}

%%%%%%%%%%%%%%%%%%%%%% DEFINE COLORS
\usepackage{color}
\definecolor{UniBlue}{RGB}{83,121,170}

%%%%%%%%%%%%%%%%%%%%%% CHAPTER & SECTION TITLES
\renewcommand{\chaptermark}[1]{\markboth{\chaptername\ \thechapter.\ #1}{}}
\renewcommand{\sectionmark}[1]{\markright{\thesection\ #1}}

\newcommand{\en}{\selectlanguage{english}}
\newcommand{\gr}{\selectlanguage{greek}}

\newcommand{\HRule}{\rule{\linewidth}{0.08mm}} % horizontal lines

\captionsetup[figure]{name=Γράφημα,singlelinecheck=off,labelfont=bf,justification=justified,singlelinecheck=false}
\captionsetup[table]{name=ΠΙΝΑΚΑΣ,labelformat=simple, labelsep=period,labelfont=bf,textfont={it},justification=justified}

\usepackage{eso-pic}
\newcommand\BackgroundPic{%
\put(0,0){%
\parbox[b][\paperheight]{\paperwidth}{%
\vfill
\centering
\includegraphics[width=\paperwidth,height=\paperheight,%
keepaspectratio]{upatrasLogo-stamp.jpg}%
\vfill
}}}


%%%%%%%%%%%%%%%%%%%%%% DOCUMENT BEGINS!!!!!!!!! %%%%%%%%%%%%%%%%%%%%%%

\begin{document}

% greek is the default language. Change with \en if you would like
% to add english text
\gr
	
\begin{titlepage}
	\begin{center}
	   
	   	   
      {\large Πανεπιστήμιο Πατρών}
	   \begin{figure}[htb!]
		\centering
		\includegraphics[scale=0.5]{upatrasLogo-stamp.jpg}{}
   	   \end{figure}
   	   
	
	
\hfill \break


%----------------------------------------------------------------------------------------
%	TITLE SECTION
%---------------------------------------------------------------
	   
		
		\HRule \\[0.1cm]
        {\Large  Τίτλος διπλωματικής εργασίας {\en Thesis title}} % Title of your document
       \HRule \\[0.1cm]
			
		

	\hfill \break
    \textbf{Όνομα Επώνυμο}
	
	\hfill \break
	\hfill \break
	 
	
	
	
	\textbf{\footnotesize «Εφαρμοσμένη Οικονομική και Ανάλυση Δεδομένων» \\
	Τμήμα Οικονομικών Επιστημών \\ 
	Σχολή Οικονομικών Επιστημών και Διοίκησης Επιχειρήσεων } 


\hfill \break
\hfill \break
\bigskip 

{\footnotesize Διατριβή που υπεβλήθη για τη μερική ικανοποίηση των
απαιτήσεων για την απόκτηση Μεταπτυχιακού Διπλώματος Ειδίκευσης}


\hfill \break
\hfill \break
	
	
% Date should appear on the bottom of the title page. Adjust
% line-breaks approproately if title is too large resulting
% in date to overflow into next page
{Μήνας  Έτος}\\
\end{center}

\end{titlepage}
	





\newpage
\thispagestyle{empty}
Πανεπιστήμιο Πατρών, Τμήμα Οικονομικών Επιστημών

Όνομα Συγγραφέα

{\copyright} $20\#\#$ $-$ Με την επιφύλαξη παντός δικαιώματος


\newpage
\thispagestyle{empty}
\begin{center}
	\textbf{Τριμελής Επιτροπή Επίβλεψης διπλωματικής εργασίας}
\end{center}
\bigskip \bigskip
\begin{tabular}{ l l l }
	\textbf{Επιβλέπων/πουσα}: & Όνομα Επώνυμο  & Αναπληρωτής Καθηγητής\\\\
	\textbf{Μέλος Επιτροπής:}  &  Όνομα Επώνυμο & Αναπληρώτρια Καθηγήτρια\\\\
	\textbf{Μέλος Επιτροπής:}  &  Όνομα Επώνυμο & Επίκουρη Καθηγήτρια
\end{tabular}
\\

\bigskip \bigskip \bigskip \bigskip \bigskip


Η παρούσα διατριβή με τίτλο


\smallskip
\begin{center}
\textit{\guillemotleft Ο τίτλος της Διπλωματικής Εργασίας\guillemotright}
\end{center}

\smallskip

εκπονήθηκε από τον/την \textbf{Όνομα Επώνυμο, Α.Μ 0182015}, για τη μερική ικανοποίηση των απαιτήσεων για την απόκτηση Μεταπτυχιακού Διπλώματος Ειδίκευσης στην \textit{\guillemotleft Εφαρμοσμένη Οικονομική και Ανάλυση Δεδομένων\guillemotright} από το Πανεπιστήμιο Πατρών και εγκρίθηκε από τα μέλη της τριμελούς επιβλέπουσας επιτροπής.

\newpage
\thispagestyle{empty}
\bigskip
\textsl{Θα ήθελα να αφιερώσω τη διπλωματική μου εργασία ...}
\newpage

\newpage
\thispagestyle{empty}
\begin{center}
	\textbf{Ευχαριστίες}\\
\end{center}
{Θα ήθελα να ευχαριστήσω ...}\\


\newpage
\pagenumbering{roman}
\begin{center}
	\textbf{Περίληψη}\\
\end{center}
Στόχος της παρούσας διπλωματικής εργασίας είναι ... 
\bigskip
\bigskip
\\
\textsl{Λέξεις κλειδιά}: {λέξη κλειδί 1, λέξη κλειδί 2, λέξη κλειδί 3, ...}


\newpage
{\la

\begin{center}
	\textbf{Summary}\\
\end{center}
In this dissertation, we examine ... 
\bigskip
\bigskip
\\
\textsl{Keywords}: {Keyword 1, Keyword 2, Keyword 3, ...}

}


\newpage
\tableofcontents
\clearpage

\renewcommand\listfigurename{}\textbf{Λίστα Σχημάτων}
\listoffigures
\clearpage

\renewcommand\listtablename{}\textbf{Λίστα Πινάκων}
\listoftables
\clearpage

\pagenumbering{arabic}

\newpage

%%%%%%%%%%%%%%%%%%%%%% LINE SPACING FOR MAIN TEXT %%%%%%%%%%%%%%%%%%%%%%
\doublespacing
\chapter{Εισαγωγή}
Τέτοιου είδους \textcolor{red}{διαγράμματα} με τον \colorbox{UniBlue}{υποδείκτη} να εμφανίζεται (μετράται) στον οριζόντιο άξονα και την τιμή της μεταβλητής να εμφανίζεται στον κάθετο άξονα είναι εξαιρετικά συνηθισμένα στα δεδομένα χρονοσειρών (τα οποία θα συζητήσουμε στην επόμενη ενότητα) αφού τότε ο υποδείκτης, που μετρά τον χρόνο, απεικονίζεται στον οριζόντιο άξονα και αντιστοιχεί σε μία «φυσιολογική» χρονολογική διάταξη. \textbf{Στα διαστρωματικά δεδομένα} ο υποδείκτης $i$ δεν μεταφέρει \textit{... κάποια πληροφορία σχετικά με τη διάταξη των παρατηρήσεων αφού έχουν ληφθεί...} με τυχαία δειγματοληψία και έτσι τέτοιου τύπου διαγράμματα έχουν λιγότερη σημασία. Εδώ χρησιμοποιούνται τα παραπάνω διαγράμματα ώστε να δώσουν την «αίσθηση της εμφάνισης» δεδομένων που ελήφθησαν από τυχαία δειγματοληψία (παρατηρήσεις τυχαίων ανεξάρτητων μεταβλητών).

Τέτοιου είδους διαγράμματα με τον υποδείκτη να εμφανίζεται (μετράται) στον οριζόντιο άξονα και την τιμή της μεταβλητής να εμφανίζεται στον κάθετο άξονα είναι εξαιρετικά συνηθισμένα στα δεδομένα χρονοσειρών (τα οποία θα συζητήσουμε στην επόμενη ενότητα) αφού τότε ο υποδείκτης, που μετρά τον χρόνο, απεικονίζεται στον οριζόντιο άξονα και αντιστοιχεί σε μία «φυσιολογική» χρονολογική διάταξη. Στα διαστρωματικά δεδομένα ο υποδείκτης $i$ δεν μεταφέρει κάποια πληροφορία σχετικά με τη διάταξη των παρατηρήσεων αφού έχουν ληφθεί με τυχαία δειγματοληψία και έτσι τέτοιου τύπου διαγράμματα έχουν λιγότερη σημασία. Εδώ χρησιμοποιούνται τα παραπάνω διαγράμματα ώστε να δώσουν την «αίσθηση της εμφάνισης» δεδομένων που ελήφθησαν από τυχαία δειγματοληψία (παρατηρήσεις τυχαίων ανεξάρτητων μεταβλητών).

Τέτοιου είδους διαγράμματα με τον υποδείκτη να εμφανίζεται (μετράται) στον οριζόντιο άξονα και την τιμή της μεταβλητής να εμφανίζεται στον κάθετο άξονα είναι εξαιρετικά συνηθισμένα στα δεδομένα χρονοσειρών (τα οποία θα συζητήσουμε στην επόμενη ενότητα) αφού τότε ο υποδείκτης, που μετρά τον χρόνο, απεικονίζεται στον οριζόντιο άξονα και αντιστοιχεί σε μία «φυσιολογική» χρονολογική διάταξη. Στα διαστρωματικά δεδομένα ο υποδείκτης $i$ δεν μεταφέρει κάποια πληροφορία σχετικά με τη διάταξη των παρατηρήσεων αφού έχουν ληφθεί με τυχαία δειγματοληψία και έτσι τέτοιου τύπου διαγράμματα έχουν λιγότερη σημασία. Εδώ χρησιμοποιούνται τα παραπάνω διαγράμματα ώστε να δώσουν την «αίσθηση της εμφάνισης» δεδομένων που ελήφθησαν από τυχαία δειγματοληψία (παρατηρήσεις τυχαίων ανεξάρτητων μεταβλητών).

Τέτοιου είδους διαγράμματα με τον υποδείκτη να εμφανίζεται (μετράται) στον οριζόντιο άξονα και την τιμή της μεταβλητής να εμφανίζεται στον κάθετο άξονα είναι εξαιρετικά συνηθισμένα στα δεδομένα χρονοσειρών (τα οποία θα συζητήσουμε στην επόμενη ενότητα) αφού τότε ο υποδείκτης, που μετρά τον χρόνο, απεικονίζεται στον οριζόντιο άξονα και αντιστοιχεί σε μία «φυσιολογική» χρονολογική διάταξη. Στα διαστρωματικά δεδομένα ο υποδείκτης $i$ δεν μεταφέρει κάποια πληροφορία σχετικά με τη διάταξη των παρατηρήσεων αφού έχουν ληφθεί με τυχαία δειγματοληψία και έτσι τέτοιου τύπου διαγράμματα έχουν λιγότερη σημασία.

\begin{figure}[!ht]
	\centering
	\includegraphics[width=6cm,height=6cm]{cossin.jpg}
	\caption{Ένα πρώτο γράφημα}
	\label{graf1}
\end{figure}

Τέτοιου είδους διαγράμματα με τον υποδείκτη να εμφανίζεται (μετράται) στον οριζόντιο άξονα και την τιμή της μεταβλητής να εμφανίζεται στον κάθετο άξονα είναι εξαιρετικά συνηθισμένα στα δεδομένα χρονοσειρών (τα οποία θα συζητήσουμε στην επόμενη ενότητα) αφού τότε ο υποδείκτης, που μετρά τον χρόνο, απεικονίζεται στον οριζόντιο άξονα και αντιστοιχεί σε μία «φυσιολογική» χρονολογική διάταξη. Στα διαστρωματικά δεδομένα ο υποδείκτης $i$ δεν μεταφέρει κάποια πληροφορία σχετικά με τη διάταξη των παρατηρήσεων αφού έχουν ληφθεί με τυχαία δειγματοληψία και έτσι τέτοιου τύπου διαγράμματα έχουν λιγότερη σημασία. Εδώ χρησιμοποιούνται τα παραπάνω διαγράμματα ώστε να δώσουν την «αίσθηση της εμφάνισης» δεδομένων που ελήφθησαν από τυχαία δειγματοληψία (παρατηρήσεις τυχαίων ανεξάρτητων μεταβλητών).

Τέτοιου είδους διαγράμματα με τον υποδείκτη να εμφανίζεται (μετράται) στον οριζόντιο άξονα και την τιμή της μεταβλητής να εμφανίζεται στον κάθετο άξονα είναι εξαιρετικά συνηθισμένα στα δεδομένα χρονοσειρών (τα οποία θα συζητήσουμε στην επόμενη ενότητα) αφού τότε ο υποδείκτης, που μετρά τον χρόνο, απεικονίζεται στον οριζόντιο άξονα και αντιστοιχεί σε μία «φυσιολογική» χρονολογική διάταξη.

\newpage
\chapter{Επισκόπηση βιβλιογραφίας}
\section{Το πρώτο υπόδειγμα}
Τέτοιου είδους διαγράμματα με τον υποδείκτη να εμφανίζεται (μετράται) στον οριζόντιο άξονα και την τιμή της μεταβλητής να εμφανίζεται στον κάθετο άξονα είναι εξαιρετικά συνηθισμένα στα δεδομένα χρονοσειρών (τα οποία θα συζητήσουμε στην επόμενη ενότητα) αφού τότε ο υποδείκτης, που μετρά τον χρόνο, απεικονίζεται στον οριζόντιο άξονα και αντιστοιχεί σε μία «φυσιολογική» χρονολογική διάταξη.
\begin{description}
	\item[Πρώτο σημείο:] Στα διαστρωματικά δεδομένα ο υποδείκτης $i$ δεν μεταφέρει κάποια πληροφορία
	\item[Δεύτερο σημείο:] σχετικά με τη διάταξη των παρατηρήσεων αφού έχουν ληφθεί με τυχαία δειγματοληψία
	\item[Τρίτο σημείο:] και έτσι τέτοιου τύπου διαγράμματα έχουν λιγότερη σημασία κλπ \ldots
\end{description}
Τέτοιου είδους διαγράμματα με τον υποδείκτη να εμφανίζεται (μετράται) στον οριζόντιο άξονα και την τιμή της μεταβλητής να εμφανίζεται στον κάθετο άξονα είναι εξαιρετικά συνηθισμένα στα δεδομένα χρονοσειρών (τα οποία θα συζητήσουμε στην επόμενη ενότητα) αφού τότε ο υποδείκτης, που μετρά τον χρόνο, απεικονίζεται στον οριζόντιο άξονα και αντιστοιχεί σε μία «φυσιολογική» χρονολογική διάταξη. Στα διαστρωματικά δεδομένα ο υποδείκτης $i$ δεν μεταφέρει κάποια πληροφορία σχετικά με τη διάταξη των παρατηρήσεων αφού έχουν ληφθεί με τυχαία δειγματοληψία και έτσι τέτοιου τύπου διαγράμματα έχουν λιγότερη σημασία. Εδώ χρησιμοποιούνται τα παραπάνω διαγράμματα ώστε να δώσουν την «αίσθηση της εμφάνισης» δεδομένων που ελήφθησαν από τυχαία δειγματοληψία (παρατηρήσεις τυχαίων ανεξάρτητων μεταβλητών).
\subsection{Επέκτασεις του πρώτου υποδείγματος}
Τέτοιου είδους διαγράμματα με τον υποδείκτη να εμφανίζεται (μετράται) στον οριζόντιο άξονα και την τιμή της μεταβλητής να εμφανίζεται στον κάθετο άξονα είναι εξαιρετικά συνηθισμένα στα δεδομένα χρονοσειρών (τα οποία θα συζητήσουμε στην επόμενη ενότητα) αφού τότε ο υποδείκτης, που μετρά τον χρόνο, απεικονίζεται στον οριζόντιο άξονα και αντιστοιχεί σε μία «φυσιολογική» χρονολογική διάταξη.
\begin{itemize}
	\item Στα διαστρωματικά δεδομένα ο υποδείκτης $i$ δεν μεταφέρει κάποια πληροφορία σχετικά με τη διάταξη των παρατηρήσεων αφού έχουν ληφθεί με τυχαία δειγματοληψία και έτσι τέτοιου τύπου διαγράμματα έχουν λιγότερη σημασία
	\item Εδώ χρησιμοποιούνται τα παραπάνω διαγράμματα ώστε να δώσουν την «αίσθηση της εμφάνισης» δεδομένων που ελήφθησαν από τυχαία δειγματοληψία (παρατηρήσεις τυχαίων ανεξάρτητων μεταβλητών).
\end{itemize}

Τέτοιου είδους διαγράμματα με τον υποδείκτη να εμφανίζεται (μετράται) στον οριζόντιο άξονα και την τιμή της μεταβλητής να εμφανίζεται στον κάθετο άξονα είναι εξαιρετικά συνηθισμένα στα δεδομένα χρονοσειρών (τα οποία θα συζητήσουμε στην επόμενη ενότητα)
\begin{enumerate}
	\item Στα διαστρωματικά δεδομένα ο υποδείκτης $i$ δεν μεταφέρει κάποια πληροφορία σχετικά με τη διάταξη των παρατηρήσεων αφού έχουν ληφθεί με τυχαία δειγματοληψία
	\item και έτσι τέτοιου τύπου διαγράμματα έχουν λιγότερη σημασία
    \item Εδώ χρησιμοποιούνται τα παραπάνω διαγράμματα ώστε να δώσουν την «αίσθηση της εμφάνισης» δεδομένων
	\item που ελήφθησαν από τυχαία δειγματοληψία (παρατηρήσεις τυχαίων ανεξάρτητων μεταβλητών).
\end{enumerate} 

Στα διαστρωματικά δεδομένα ο υποδείκτης $i$ δεν μεταφέρει κάποια πληροφορία σχετικά με τη διάταξη των παρατηρήσεων αφού έχουν ληφθεί με τυχαία δειγματοληψία και έτσι τέτοιου τύπου διαγράμματα έχουν λιγότερη σημασία. Εδώ χρησιμοποιούνται τα παραπάνω διαγράμματα ώστε να δώσουν την «αίσθηση της εμφάνισης» δεδομένων που ελήφθησαν από τυχαία δειγματοληψία (παρατηρήσεις τυχαίων ανεξάρτητων μεταβλητών). Η παρακάτω εξίσωση
\begin{equation}
\dfrac{\sum \limits_{i=min}^{max} \alpha_{i}}{\sum \limits_{j=min}^{Ν} \beta_{j}^{2}}
\label{2.kati}
\end{equation}
περιγράφει ένα απλό κλάσμα. Η εξίσωση (\ref{2.kati}) είναι ενδεικτική.

Η παρακάτω εξίσωση δεν έχει αρίθμηση
\begin{equation*}
Y_{i}=\alpha+\beta X_{i}+u_{i}
\end{equation*}
όπως είναι εμφανές.

Τέτοιου είδους διαγράμματα με τον υποδείκτη να εμφανίζεται (μετράται) στον οριζόντιο άξονα και την τιμή της μεταβλητής να εμφανίζεται στον κάθετο άξονα είναι εξαιρετικά συνηθισμένα στα δεδομένα χρονοσειρών (τα οποία θα συζητήσουμε στην επόμενη ενότητα) αφού τότε ο υποδείκτης, που μετρά τον χρόνο, απεικονίζεται στον οριζόντιο άξονα και αντιστοιχεί σε μία «φυσιολογική» χρονολογική διάταξη.

Στα διαστρωματικά δεδομένα ο υποδείκτης $i$ δεν μεταφέρει κάποια πληροφορία σχετικά με τη διάταξη των παρατηρήσεων αφού έχουν ληφθεί με τυχαία δειγματοληψία και έτσι τέτοιου τύπου διαγράμματα έχουν λιγότερη σημασία. Εδώ χρησιμοποιούνται τα παραπάνω διαγράμματα ώστε να δώσουν την «αίσθηση της εμφάνισης» δεδομένων που ελήφθησαν από τυχαία δειγματοληψία (παρατηρήσεις τυχαίων ανεξάρτητων μεταβλητών).

\subsection{Άλλες επεκτάσεις του πρώτου υποδείγματος}
Τέτοιου είδους διαγράμματα με τον υποδείκτη να εμφανίζεται (μετράται) στον οριζόντιο άξονα και την τιμή της μεταβλητής να εμφανίζεται στον κάθετο άξονα είναι εξαιρετικά συνηθισμένα στα δεδομένα χρονοσειρών (τα οποία θα συζητήσουμε στην επόμενη ενότητα) αφού τότε ο υποδείκτης, που μετρά τον χρόνο, απεικονίζεται στον οριζόντιο άξονα και αντιστοιχεί σε μία «φυσιολογική» χρονολογική διάταξη. Στα διαστρωματικά δεδομένα ο υποδείκτης $i$ δεν μεταφέρει κάποια πληροφορία σχετικά με τη διάταξη των παρατηρήσεων αφού έχουν ληφθεί με τυχαία δειγματοληψία και έτσι τέτοιου τύπου διαγράμματα έχουν λιγότερη σημασία. Εδώ χρησιμοποιούνται τα παραπάνω διαγράμματα ώστε να δώσουν την «αίσθηση της εμφάνισης» δεδομένων που ελήφθησαν από τυχαία δειγματοληψία (παρατηρήσεις τυχαίων ανεξάρτητων μεταβλητών).

Τέτοιου είδους διαγράμματα με τον υποδείκτη να εμφανίζεται (μετράται) στον οριζόντιο άξονα και την τιμή της μεταβλητής να εμφανίζεται στον κάθετο άξονα είναι εξαιρετικά συνηθισμένα στα δεδομένα χρονοσειρών (τα οποία θα συζητήσουμε στην επόμενη ενότητα) αφού τότε ο υποδείκτης, που μετρά τον χρόνο, απεικονίζεται στον οριζόντιο άξονα και αντιστοιχεί σε μία «φυσιολογική» χρονολογική διάταξη. Στα διαστρωματικά δεδομένα ο υποδείκτης $i$ δεν μεταφέρει κάποια πληροφορία σχετικά με τη διάταξη των παρατηρήσεων αφού έχουν ληφθεί με τυχαία δειγματοληψία και έτσι τέτοιου τύπου διαγράμματα έχουν λιγότερη σημασία. Εδώ χρησιμοποιούνται τα παραπάνω διαγράμματα ώστε να δώσουν την «αίσθηση της εμφάνισης» δεδομένων που ελήφθησαν από τυχαία δειγματοληψία (παρατηρήσεις τυχαίων ανεξάρτητων μεταβλητών).
\subsubsection{Επιπλέον παρατηρήσεις: πρώτο μέρος}
Τέτοιου είδους διαγράμματα με τον υποδείκτη να εμφανίζεται (μετράται) στον οριζόντιο άξονα και την τιμή της μεταβλητής να εμφανίζεται στον κάθετο άξονα είναι εξαιρετικά συνηθισμένα στα δεδομένα χρονοσειρών (τα οποία θα συζητήσουμε στην επόμενη ενότητα) αφού τότε ο υποδείκτης, που μετρά τον χρόνο, απεικονίζεται στον οριζόντιο άξονα και αντιστοιχεί σε μία «φυσιολογική» χρονολογική διάταξη. Στα διαστρωματικά δεδομένα ο υποδείκτης $i$ δεν μεταφέρει κάποια πληροφορία σχετικά με τη διάταξη των παρατηρήσεων αφού έχουν ληφθεί με τυχαία δειγματοληψία και έτσι τέτοιου τύπου διαγράμματα έχουν λιγότερη σημασία. Εδώ χρησιμοποιούνται τα παραπάνω διαγράμματα ώστε να δώσουν την «αίσθηση της εμφάνισης» δεδομένων που ελήφθησαν από τυχαία δειγματοληψία (παρατηρήσεις τυχαίων ανεξάρτητων μεταβλητών).
\subsubsection{Επιπλέον παρατηρήσεις: δεύτερο μέρος}
Τέτοιου είδους διαγράμματα με τον υποδείκτη να εμφανίζεται (μετράται) στον οριζόντιο άξονα και την τιμή της μεταβλητής να εμφανίζεται στον κάθετο άξονα είναι εξαιρετικά συνηθισμένα στα δεδομένα χρονοσειρών (τα οποία θα συζητήσουμε στην επόμενη ενότητα) αφού τότε ο υποδείκτης, που μετρά τον χρόνο, απεικονίζεται στον οριζόντιο άξονα και αντιστοιχεί σε μία «φυσιολογική» χρονολογική διάταξη.

\begin{figure}[!ht]
	\centering
	\includegraphics[width=6cm,height=6cm]{MU3BY800.png}
	\caption{Στο γράφημα εμφανίζονται τρεις τυχαία σχεδιασμένες γραμμές που αντιστοιχούν σε τρεις διαφορετικές «εκτιμήσεις» των $\alpha,\beta$.}
	\label{sxima2}
\end{figure}

Στα διαστρωματικά δεδομένα ο υποδείκτης $i$ δεν μεταφέρει κάποια πληροφορία σχετικά με τη διάταξη των παρατηρήσεων αφού έχουν ληφθεί με τυχαία δειγματοληψία και έτσι τέτοιου τύπου διαγράμματα έχουν λιγότερη σημασία. Εδώ χρησιμοποιούνται τα παραπάνω διαγράμματα ώστε να δώσουν την «αίσθηση της εμφάνισης» δεδομένων που ελήφθησαν από τυχαία δειγματοληψία (παρατηρήσεις τυχαίων ανεξάρτητων μεταβλητών).

\section{Το δεύτερο υπόδειγμα}
Τέτοιου είδους διαγράμματα με τον υποδείκτη να εμφανίζεται (μετράται) στον οριζόντιο άξονα και την τιμή της μεταβλητής να εμφανίζεται στον κάθετο άξονα είναι εξαιρετικά συνηθισμένα στα δεδομένα χρονοσειρών (τα οποία θα συζητήσουμε στην επόμενη ενότητα) αφού τότε ο υποδείκτης, που μετρά τον χρόνο, απεικονίζεται στον οριζόντιο άξονα και αντιστοιχεί σε μία «φυσιολογική» χρονολογική διάταξη. Στα διαστρωματικά δεδομένα ο υποδείκτης $i$ δεν μεταφέρει κάποια πληροφορία σχετικά με τη διάταξη των παρατηρήσεων αφού έχουν ληφθεί με τυχαία δειγματοληψία και έτσι τέτοιου τύπου διαγράμματα έχουν λιγότερη σημασία. Εδώ χρησιμοποιούνται τα παραπάνω διαγράμματα ώστε να δώσουν την «αίσθηση της εμφάνισης» δεδομένων που ελήφθησαν από τυχαία δειγματοληψία (παρατηρήσεις τυχαίων ανεξάρτητων μεταβλητών).

Τέτοιου είδους διαγράμματα με τον υποδείκτη να εμφανίζεται (μετράται) στον οριζόντιο άξονα και την τιμή της μεταβλητής να εμφανίζεται στον κάθετο άξονα είναι εξαιρετικά συνηθισμένα στα δεδομένα χρονοσειρών (τα οποία θα συζητήσουμε στην επόμενη ενότητα) αφού τότε ο υποδείκτης, που μετρά τον χρόνο, απεικονίζεται στον οριζόντιο άξονα και αντιστοιχεί σε μία «φυσιολογική» χρονολογική διάταξη. Στα διαστρωματικά δεδομένα ο υποδείκτης $i$ δεν μεταφέρει κάποια πληροφορία σχετικά με τη διάταξη των παρατηρήσεων αφού έχουν ληφθεί με τυχαία δειγματοληψία και έτσι τέτοιου τύπου διαγράμματα έχουν λιγότερη σημασία. Εδώ χρησιμοποιούνται τα παραπάνω διαγράμματα ώστε να δώσουν την «αίσθηση της εμφάνισης» δεδομένων που ελήφθησαν από τυχαία δειγματοληψία (παρατηρήσεις τυχαίων ανεξάρτητων μεταβλητών).

\newpage
\chapter{Συμπεράσματα}

Ευκαιρία να παρουσιαστεί και ένας πίνακας αποτελεσμάτων όπως ο παρακάτω

\begin{table}[!ht]

	\centering % used for centering table
	\begin{tabular}{c c c c} % centered columns (4 columns)
		\hline\hline %inserts double horizontal lines
		Περίπτωση & Μέθοδος 1 & Μέθοδος 2 & Μέθοδος 3 \\ [0.75ex]
		\hline % inserts single horizontal line
		1 \vline & 50 & 837 & 970 \\[0.75ex] % inserting body of the table
		2 \vline & 47 & 877 & 230 \\[0.75ex]
		3 \vline & 31 & 25 & 415 \\[0.75ex]
		$\hat{\beta}$ \vline & 35 & 144 & 2356 \\[0.75ex]
		5 \vline & 45 & 300 & 556 \\ [1ex] % [1ex] adds vertical space
		\hline %inserts single line
	\end{tabular}
	\caption{Μη-γραμμικό υπόδειγμα εκτιμημένο με τη μέθοδο των γενικευμένων ροπών.} % title of Table
	\label{tab1} % is used to refer this table in the text
\end{table}

Τέτοιου είδους διαγράμματα και πίνακες όπως ο πίνακας (\ref{tab1}) με τον υποδείκτη να εμφανίζεται (μετράται) στον οριζόντιο άξονα και την τιμή της μεταβλητής να εμφανίζεται στον κάθετο άξονα είναι εξαιρετικά συνηθισμένα στα δεδομένα χρονοσειρών (τα οποία θα συζητήσουμε στην επόμενη ενότητα) αφού τότε ο υποδείκτης, που μετρά τον χρόνο, απεικονίζεται στον οριζόντιο άξονα και αντιστοιχεί σε μία «φυσιολογική» χρονολογική διάταξη.

Στα διαστρωματικά δεδομένα ο υποδείκτης $i$ δεν μεταφέρει κάποια πληροφορία σχετικά με τη διάταξη των παρατηρήσεων αφού έχουν ληφθεί με τυχαία δειγματοληψία και έτσι τέτοιου τύπου διαγράμματα έχουν λιγότερη σημασία. Εδώ χρησιμοποιούνται τα παραπάνω διαγράμματα ώστε να δώσουν την «αίσθηση της εμφάνισης» δεδομένων που ελήφθησαν από τυχαία δειγματοληψία (παρατηρήσεις τυχαίων ανεξάρτητων μεταβλητών).

Στα διαστρωματικά δεδομένα ο υποδείκτης $i$ δεν μεταφέρει κάποια πληροφορία σχετικά με τη διάταξη των παρατηρήσεων αφού έχουν ληφθεί με τυχαία δειγματοληψία και έτσι τέτοιου τύπου διαγράμματα έχουν λιγότερη σημασία. Εδώ χρησιμοποιούνται τα παραπάνω διαγράμματα ώστε να δώσουν την «αίσθηση της εμφάνισης» δεδομένων που ελήφθησαν από τυχαία δειγματοληψία (παρατηρήσεις τυχαίων ανεξάρτητων μεταβλητών).

\newpage
\addcontentsline{toc}{chapter}{Βιβλιογραφία}

{\LARGE \textbf{Βιβλιογραφία}}

\bigskip
{\large \textbf{Ελληνική}}
\begin{list}{ }{\itemsep3pt}
	\item Τζελέπης, Δ., (2007). Μαθήματα λογιστικής {\la I}. Εκδόσεις Ανώνυμες, Πάτρα
\end{list}

\bigskip
{\large \textbf{Αγγλική}}
\begin{list}{ }{\itemsep3pt}
	\item {\la Giannakopoulos, N., \& Venetis, I., (2015). Some publication on economics. \emph{Journal of Publications}, 15, 1-31}
	\item {\la Tzagkarakis, E., Giannakopoulos, N., Venetis, I., \& Tzelepis, D., (1986). Many authors here. \emph{Journal of Many Authors}, 25, 10-23}
	\item {\la Tzelepis, D., (2012). On my lessons. Some University Press, Patra, Greece, 2nd Edition}
	\item {\la Venetis, I., (2003a). Proof of my theorem. Preprint, available at \url{http://www.some.site.edu/mypapers/prooftheorem.pdf}}
	\item {\la Venetis, I., (2003b). Yet another proof. \emph{Journal of Applied Tutorials}, 43(2), 444-455}
\end{list}






%%%%%%%%%%%%%%%%%%%%%%%%%%%%%%%%%%%%%%%%%%%%%%%%%%%%%%%%%%%%%
% Appendices
%%%%%%%%%%%%%%%%%%%%%%%%%%%%%%%%%%%%%%%%%%%%%%%%%%%%%%%%%%%%%

\newpage
\appendix



\chapter{}
Περιεχόμενο παραρτήματος


\chapter{}
Άλλο παράρτημα 





\end{document}
